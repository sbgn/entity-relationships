% =============================================================================
% preface
% =============================================================================

\chapter{Preface}

The present document describes \SBGNERLone. The \chap{glyph} on page ~\pageref{chp:glyph} provides a catalog of the graphical symbols (glyphs) available for representing entities nodes and relationships in \ER diagrams. It is targeted to all audiences, wanting to generate or interpret \ER diagrams, and is (hopefully) not too technical. In \chap{grammar} beginning on page~\pageref{chp:grammar}, we describe the rules for combining these glyphs into a legal SBGN \ER, and in \chap{layout} beginning on page~\pageref{chp:layout}, we describe requirements and guidelines for the way that diagrams are visually organized. Those chapters are targeted to people wishing to generate SBGN diagrams, whether manually or automatically, and are more technical.

\section*{Acknowledgements}

The authors are grateful to all the attendees of the SBGN meetings, as well as to the subscribers of the \mailto{sbgn-discuss@sbgn.org} mailing list.  

% The authors would like to acknowledge especially the help of X, Y and Z. A more comprehensive list of people involved in SBGN development is available in the appendix~\ref{sec:acknowledgments}.
% 
The development of SBGN was mainly supported by a grant from the Japanese \emph{New Energy and Industrial Technology Development Organization} (NEDO, \url{http://www.nedo.go.jp/}).  The \emph{Okinawa Institute of Science and Technology} (OIST, \url{http://www.oist.jp/}), the \emph{AIST Computational Biology Research Center} (AIST CBRC, \url{http://www.cbrc.jp/index.eng.html}) the British \emph{Biotechnology and Biological Sciences Research Council} (BBSRC, \url{http://www.bbsrc.ac.uk/}) through a Japan Partnering Award, the European Media Laboratory (EML Research gGmbH, \url{http://www.eml-r.org/}), the Beckman Institute at the California Institute of Technology (\url{http://bnmc.caltech.edu}) and the Auckland Bioengineering Institute (\url{http://www.bioeng.auckland.ac.nz/}) provided additional support for SBGN workshops.

\section*{Notes on typographical conventions}

The concept represented by a glyph is written using a normal font, while a \glyph{glyph} means the SBGN visual representation of the concept. Note on the color code: \textcolor{blue}{The glyphs that have been thorougly discussed, and are considered frozen, are represented in blue}. \textcolor{ForestGreen}{The glyphs that have been thorougly discussed, but either are still posing problems, or have an unclear semantics are represented in green}. \textcolor{red}{The glyphs that have been proposed but for which in-depth discussion is yet to come are represented in red}.



% The following is for [X]Emacs users.  Please leave in place.
% Local Variables:
% TeX-master: "../sbgn_ER-level1"
% End:
