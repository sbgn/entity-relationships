% =============================================================================
% front-page
% =============================================================================

\begin{titlepage}

\vspace*{0.75in}

\begin{center}

  \textbf{\sffamily\bfseries\huge
    Systems Biology Graphical Notation:\\[0.3em]
    Entity Relationship language Level 1}

\vspace*{0.5in}

\large
Draft of %\sbgndate\\[0.25in]
\today\\[0.25in]

\cornersize{0.3}\ovalbox{\begin{minipage}{4.9in}\color{DarkRed}
Disclaimer: This is a working draft of the SBGN Entity Relationship
Level~1 specification.  It is not a normative document.
\end{minipage}}

\vspace{0.5in}

\textbf{\sffamily Editors}:\\[7pt]
\begin{tabular}{l>{\hspace*{15pt}}r}
Nicolas \lenov   & \emph{EMBL European Bioinformatics Institute, UK}\\
Stuart Moodie    & \emph{CSBE, University of Edinburgh, UK}\\
Anatoly Sorokin  & \emph{University of Edinburgh, UK}\\
Falk Schreiber	 & \emph{IPK Gatersleben \& University of Halle, DE}\\
Huaiyu Mi        & \emph{SRI International, USA}\\
\end{tabular}
 
\vfill

\normalsize
\begin{minipage}{5in}
  \emph{To discuss any aspect of SBGN, please send your messages
    to the mailing list \mailto{sbgn-discuss@sbgn.org}.  To get
    subscribed to the mailing list or to contact us directly,
    please write to \mailto{sbgn-editors@lists.sourceforge.net}. Bug reports and specific comments about the specification should be entered in the issue tracker \url{https://sourceforge.net/tracker/admin/?atid=1170625&group_id=178553}.}
\end{minipage}

\vfill


\centerline{\includegraphics[width=1.25in]{\SBGNLogoFile}}


\end{center}

\end{titlepage}

% The title page is considered unnumbered and the next page after this
% starts with the page number 1 (actually, i), but the duplication of page
% number 1 confuses hyperref and leads to the following latex warning:
%
%   "pdfTeX warning (ext4): destination with the same identifier
%   (name{page.1}) has been already used, duplicate ignored"
%
% The following change makes the title page have page number 1 and the next
% page after that it becomes page ii.  This is unorthodox, but seems not
% completely unreasonable, and it avoids the confusing warning above.

\setcounter{page}{2}
