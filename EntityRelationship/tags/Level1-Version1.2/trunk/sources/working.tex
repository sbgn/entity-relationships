% !TEX root = ../sbgn_ER-level1.tex
\chapter{Issues postponed to future levels}\label{sec:postponed}

\section{Generics and instances}\label{sec:instances}

In \SBGNERLone, an entity is represented only once. One cannot explicitly represent different instances of the ``same'' entity. Several instances can be inferred from relationships acting in \glyph{trans}. However, one cannot generally express the fact that several relationships involving the same entity actually involve the same, or different, instances of this entity. This problem is tied to the problem of generics. Indeed, if one discriminate between classes of instances, how can one represent, in the same map, the generic entity?


\section{Groups}\label{sec:groups}

The \SBGNERLone needs a mechanism to "tag" an SBGN glyph as belonging to one or several groups. Those groups would correspond for instance to "pathways" or "metabolic network", glyphs associated with a disease or biological function. A group does not affect the syntax of the map, but is merely a multi-glyph annotation.

Groups would be a way for instance to organise nodes together in a certain subpart of the plan, or highligtht them in some way. Software would have to conserve groups, and it could thus be a way to lightly constrain the layout, without going all the way to specify position and size of the nodes. A group would not be "linked" to any node using edges, but would "contain" EPN/PN in PD, EN in ER, AN in AF (i.e. in PD and AF, a group could span several compartment). Groups could be named and could be annotated with "floating" annotations, similarly to the default compartment.

\section{Submap}\label{sec:submap}

The \SBGNPDLone has a \glyph{Submap} as placeholder for another process and is used when one wishes to hide the detail of this process from the Process Description map, but make it available to the reader as a separate related map. The similar concept could be implemented in \SBGNERLone.

\section{Spatial organisation}\label{sec:spacial}

Introduction of \glyph{Nested entities} makes representation of domains possible. The additional feature required by some users is to be able to show relation between domains. For example, promoter, coding region and terminator would be domains of the plasmid entity and it would be important to show that promoter is located before coding region. 

%\section{Synchronous events}\label{sec:syncEvents}

%In \SBGNERLone, each influence arc started from its own outcome and so asynchronous . There is a need to represent that two events happened simultaneously or caused by one outcome, for example to represent chemical reaction we need to show that destruction of substrates and creation of products happened simultaneously.