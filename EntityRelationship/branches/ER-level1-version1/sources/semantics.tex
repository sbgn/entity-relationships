%\color{red}
\section{Semantic description of \ERs}

% Add small images of the glyphs

\subsection{Statements}

An \glyph{interaction} (\sect{interaction}) linking the \glyph{interactors} $A$ and $B$ means: ``$A$ interacts with $B$''. An \glyph{outcome} on an \glyph{interaction} represents the cases when the statement is true, that is when the interaction effectively exists. If the interaction is a physical interaction between molecules, the \glyph{outcome} represents the complex resulting from the interaction. It is used as follow: ``when (or if) $A$ interacts with $B$ then \ldots''.\\[\baselineskip]

\noindent
An \glyph{assignment} (\sect{assignment}) linking a state variable value $v$ to a \glyph{state-variable} $V$ of an \glyph{entity} $E$ means: ``$v$ is assigned to $V$ of $E$'' or ``$V$ of $E$ takes the value $v$''. An \glyph{outcome} on an \glyph{assignment} represents the cases when the statement is true, that is when the variable effectively displays the value. It is used as follows: ``when (or if) $V$ of $E$ takes the value $v$ then \ldots'' or more succintly ``when (or if) $E\{V => v\}$ then \ldots''.\\[\baselineskip]

\noindent
A \glyph{phenotype} (\sect{phenotype}) $P$ means: ``$P$ exists''.\\[\baselineskip]

\subsection{Influences}

A \glyph{modulation} (\sect{modulation}) linking an \glyph{entity node} $E$ and a relationship $R$ means: ``If $E$ exists then $R$ is either reinforced or weakened''. 
%An \glyph{outcome} on a \glyph{modulation} represents the cases when the modulation effectively takes place. It is used as follow: ``when (or if) $I$ modulates $R$ then \ldots''.
\\[\baselineskip]

\noindent
A \glyph{stimulation} (\sect{stimulation}) linking an \glyph{entity node} $E$ and a relationship $R$ means: ``If $E$ exists then $R$ is reinforced'' or ``If $E$ exists then the probability of $R$ is increased''. 
%An \glyph{outcome} on a \glyph{stimulation} represents the cases when the stimulation effectively takes place. It is used as follow: ``when (or if) $I$ stimulates $R$ then \ldots''.
\\[\baselineskip]

\noindent
An \glyph{absolute stimulation} (\sect{absoluteStimulation}) linking an \glyph{entity node} $E$ and a relationship $R$ means: ``If $E$ exists then $R$ always takes place''. 
\\[\baselineskip]

\noindent
A \glyph{necessary stimulation} (\sect{necessaryStimulation}) linking an \glyph{entity node} $E$ and a relationship $R$ means: ``$R$ only takes place if $E$ exists. 
%An \glyph{outcome} on a \glyph{necessary stimulation} represents the cases when the stimulation effectively takes place. It is used as follow: ``when (or if) $I$ allows $R$ then \ldots''.
\\[\baselineskip]

\noindent
An \glyph{inhibition} (\sect{inhibition}) linking an \glyph{entity node} $E$ and a relationship $R$ means: ``If $E$ exists then $R$ is weakened'' or ``If $E$ exists then the probability of $R$ is lowered''. 
%An \glyph{outcome} on an \glyph{inhibition} represents the cases when the inhibition effectively takes place. It is used as follow: ``when (or if) $I$ inhibits $R$ then \ldots''.
\\[\baselineskip]

\noindent
An \glyph{absolute inhibition} (\sect{absoluteInhibition}) linking an \glyph{entity node} $E$ and a relationship $R$ means: ``If $E$ exists then $R$ never takes place''. 
%An \glyph{outcome} on an \glyph{absolute inhibition} represents the cases when the absolute inhibition effectively takes place. It is used as follow: ``when (or if) $I$ blocks $R$ then \ldots''.
\\[\baselineskip]

\subsection{Logical Operators}

An \glyph{and} (\sect{and}) linking several \glyph{logic arcs} originating from \glyph{entity nodes} $E_i$ and an influence $F$ means: ``if for each $i$, $E_i$ exists, then $F$''.\\[\baselineskip]

\noindent
An \glyph{or} (\sect{or}) linking several \glyph{logic arcs} originating from \glyph{entity nodes} $E_i$ and an influence $F$ means: ``if for any $i$, $E_i$ exists, then $F$''.\\[\baselineskip]

% We need to work on that. Expressed that way, it does not make sense.
\noindent
A \glyph{not} (\sect{not}) linking a \glyph{logic arc} originating from an \glyph{entity node} $E$ and an influence $F$ means: ``if $E$ does not exist, then''.\\[\baselineskip]

\noindent
A \glyph{delay} (\sect{delay}) linking a \glyph{logic arc} originating from an \glyph{entity node} $E$ and an influence $F$ means: ``If $E$ exists then $F$ takes place, but not immediately''.\\[\baselineskip]

\subsection{Cis and trans relationships}

The use of cis and trans units of information on a combination of relationships brings power and versatility to \ERs. However, the resulting semantics may be difficult to grasp. Here are the basic rules that permit to understand the graphs.

\begin{itemize}
 \item The unit of information ``cis'' or ``trans'' carried by an \glyph{interaction} refers to the \glyph{interactors} targeted by the \glyph{interaction}. 
 \item The unit of information ``cis'' or ``trans'' carried by an \glyph{influence} targeting a state variable \glyph{assignment} refers to the origin of the \glyph{influence} and to the \glyph{entity} carrying the target of the \glyph{assignment}. 
\item The unit of information ``cis'' or ``trans'' carried by an \glyph{influence} targeting another \glyph{influence} refers to the origin of the carrying \glyph{influence} and to the origin of the targeted \glyph{influence}.
 \item The unit of information ``cis'' or ``trans'' carried by an \glyph{influence} targeting an \glyph{interaction} refers to the origin of the \glyph{influence} and all the relevant \glyph{interactors} targeted by the \glyph{interaction} (see \sect{SyntacticRules}).
\end{itemize}

\subsection{(In)Validation of ER maps}

Based on the definitions above, it should be possible to use the toolkit of formal logic to analyse \ERs{}. In particular, one can envision to build truth tables describing the consequences of the existences of the various entities. Those table should point to inconsistencies leading to contradictory predicates.

% Give an example of a wrong ER map.

%\normalcolor

