%%%%%%%%%%%%%%%%%%%%%%%%%%%%%%%%%%%%%%%%%%%%%%%%%%%%%%%%%%%%%%%%%%%%%%
%%%%                   Controlled vocabularies
%%%%%%%%%%%%%%%%%%%%%%%%%%%%%%%%%%%%%%%%%%%%%%%%%%%%%%%%%%%%%%%%%%%%%%

\section{Controlled vocabularies used in \SBGNERLone}\label{sec:CVs}

Some glyphs in SBGN \ERs can contain particular kinds of textual annotations conveying information relevant to the purpose of the glyph.  These annotations are carried by \glyph{units of information} (\sect{unitInformation}) or \glyph{state variable values} (\sect{stateVariable}).

The text that appears as the unit of information decorating an entity must be prefixed with a controlled vocabulary term indicating the type of information being expressed.  The prefixes are mandatory.  Without the use of controlled vocabulary prefixes, it would be necessary to have different glyphs to indicate different classes of information; this would lead to an explosion in the number of symbols needed.

In the rest of this section, we describe the controlled vocabularies (CVs) used in \SBGNERLone.  In each case, some CV terms are predefined by SBGN, but unless otherwise noted, \emph{they are not the only terms permitted}.  Authors may use other CV values not listed here, but in such cases, they should explain the terms' meanings in a figure legend or other text accompanying the map.

\subsection{Entity material types}
\label{sec:material-types-cv}

The material type of an \glyph{Entity} indicates its chemical structure.  A list of common material types is shown in \fig{material-types-cv}, but others are possible.  The values are to be taken from the \sbo (\sbourl), specifically from the branch having identifier \sboid{SBO:0000240} (\emph{material entity}).  The labels are defined by \SBGNERLone.

\begin{figure}[h]
  \centering
  \begin{tabular}{l>{\ttfamily}ll}
    \toprule
    \textbf{Name}              & \textbf{\rmfamily Label} & \textbf{SBO term} \\
    \midrule
    Non-macromolecular ion     & mt:ion  & SBO:0000327\\
    Non-macromolecular radical & mt:rad  & SBO:0000328\\
    Ribonucleic acid           & mt:rna  & SBO:0000250\\
    Deoxribonucleic acid       & mt:dna  & SBO:0000251\\
    Protein                    & mt:prot & SBO:0000297\\
    Polysaccharide             & mt:psac & SBO:0000249\\
    \bottomrule
  \end{tabular}
  \caption{A sample of values from the \emph{material types} controlled
    vocabulary (\sect{material-types-cv}).}
  \label{fig:material-types-cv}
\end{figure}

The material types are in contrast to the \emph{conceptual types} (see below).  The distinction is that material types are about physical composition, while conceptual types are about functions.  For example, a strand of RNA is a physical artifact, but its use as messenger RNA is a function.

\subsection{Entity conceptual types}
\label{sec:conceptual-types-cv}

An \glyph{entity}'s \emph{conceptual type} indicates its function within the context of a given \ERm.  A list of common conceptual types is shown in \fig{conceptual-types-cv}, but others are possible.  The values are to be taken from the \sbo (\sbourl), specifically from the branch having identifier \sboid{SBO:0000241} (\emph{functional entity}).  The labels are defined by \SBGNERLone.

\begin{figure}[h]
  \centering
  \begin{tabular}{l>{\ttfamily}ll}
    \toprule
    \textbf{Name}              & \textbf{\rmfamily Label} & \textbf{SBO term} \\
    \midrule
    Gene                      & ct:gene   & SBO:0000243\\
    Transcription start site  & ct:tss    & SBO:0000329\\
    Gene coding region        & ct:coding & SBO:0000335\\
    Gene regulatory region    & ct:grr    & SBO:0000369\\
    Messenger RNA             & ct:mRNA   & SBO:0000278\\
    \bottomrule
  \end{tabular}
  \caption{A sample of values from the \emph{conceptual types} vocabulary
    (\sect{conceptual-types-cv}).}
  \label{fig:conceptual-types-cv}
\end{figure}

\subsection{Macromolecule covalent modifications}
\label{sec:covalent-mod-cv}

A common reason for the introduction of state variables on an entity is to allow access to the configuration of possible covalent modification sites on that entity.  For instance, a macromolecule may have one or more sites where a phosphate group may be attached; this change in the site's configuration (\ie being either phosphorylated or not) may factor into whether, and how, the entity can participate in different processes.  Being able to describe such modifications in a consistent fashion is the motivation for the existence of SBGN's covalent modifications controlled vocabulary.  

\fig{covalent-mod-cv} lists a number of common types of covalent modifications.  The most common values are defined by the \sbo in the branch having identifier \sboid{SBO:0000210} (\emph{addition} under \emph{events}$\rightarrow$\emph{reaction}$\rightarrow$\emph{biochemical reaction}$\rightarrow$\emph{conversion}$\rightarrow$\emph{addition}).  The labels shown in \fig{covalent-mod-cv} are defined by \SBGNERLone; for all other kinds of modifications not listed here, the author of an \ERm must create a new label (and should also describe the meaning of the label in a legend or text accompanying the map).

\begin{figure}[h]
  \centering
  \begin{tabular}{l>{\ttfamily}ll}
    \toprule
    \textbf{Name}   & \textbf{\rmfamily Label} & \textbf{SBO term} \\
    \midrule
    Acetylation     & Ac    & SBO:0000215\\
    Glycosylation   & G     & SBO:0000217\\
    Hydroxylation   & OH    & SBO:0000233\\
    Methylation     & Me    & SBO:0000214\\
    Myristoylation  & My    & SBO:0000219\\
    Palmytoylation  & Pa    & SBO:0000218\\
    Phosphorylation & P     & SBO:0000216\\
    Prenylation     & Pr    & SBO:0000221\\
    Protonation     & H     & SBO:0000212\\
    Sulfation       & S     & SBO:0000220\\
    Ubiquitination  & Ub    & SBO:0000224\\
    \bottomrule
  \end{tabular}
  \caption{A sample of values from the \emph{covalent modifications} vocabulary
    (\sect{covalent-mod-cv}).}
  \label{fig:covalent-mod-cv}
\end{figure}

\subsection{Miscellaneous terms}
\label{sec:miscellaneous-cv}

\SBGNERLone requires several reserved characters. A special unit of information usable on interactions describe the number of identical interactors involved. Note that the value is a unitary number, and not (for example) a range.  There is no provision in \SBGNPDLone for specifying a range in this context because it leads to problems of entity identifiability. Other reserved characters are used in state variable assignments to represent truth or falsehood. Two reserved words are used in units of information carried by relationships: cis and trans. 

\begin{table}[h]
  \centering
  \begin{tabular}{l>{\ttfamily}l>{\ttfamily}l}
    \toprule
    \textbf{Name}   & \textbf{\rmfamily Label} & \textbf{\rmfamily SBO term} \\
    \midrule
    cardinality    & \#  & SBO:0000364\\
    true           & T     & SBO:0000416\\
    false          & F     & SBO:0000417\\
    cis            & cis   & SBO:0000414\\
    trans          & trans & SBO:0000415\\
    \bottomrule
  \end{tabular}
  \caption{Miscellaneous controlled terms. For the cardinality, \texttt{\#} stands for a number, for example, ``\texttt{5}''.}
  \label{tab:cardinality-cv}
\end{table}

%\normalcolor
