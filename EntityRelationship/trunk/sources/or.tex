%%%%%%%%%%%%%%%%%%%%%%%%%%%%%%%%%%%%%%%%%%%%%%%%%%%%%%%%%%%%%%%%%%%%%%
%%                     Or
%%%%%%%%%%%%%%%%%%%%%%%%%%%%%%%%%%%%%%%%%%%%%%%%%%%%%%%%%%%%%%%%%%%%%%
\color{blue}
\subsection{Glyph: \glyph{Or}}\label{sec:or}

The glyph \glyph{or} is used to denote that any of the \glyph{EPNs} linked as input is sufficient to produce the output influence.  

\begin{glyphDescription}
 \glyphSboTerm SBO:0000174 ! or.
 \glyphOrigin One interactor (section~\ref{sec:interactors}) or logical operator (section~\ref{sec:logic}).
 \glyphTarget  One modulation (section~\ref{sec:modulation}), stimulation (section~\ref{sec:stimulation}), inhibition (section~\ref{sec:inhibition}), necessary  stimulation (section~\ref{sec:necessaryStimulation}), or absolute inhibition (section~\ref{sec:absoluteInhibition}) arc.
 \glyphNode \glyph{Not} is represented by a circle carrying the word ``OR''.
 \end{glyphDescription}

\begin{figure}[H]
  \centering
  \includegraphics[scale = 0.5]{images/or}
  \caption{The \ER glyph for \glyph{or}. Only two inputs are represented, but more would be allowed.}
  \label{fig:or}
\end{figure}


