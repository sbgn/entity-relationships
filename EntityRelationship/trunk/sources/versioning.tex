% =============================================================================
% SBGN versioning
% =============================================================================

\section{SBGN levels and versions}
\label{sec:sbgn-levels}

It was clear at the outset of SBGN development that it would be impossible to design a perfect and complete notation right from the beginning.  Apart from the prescience this would require (which, sadly, none of the authors possess), it also would likely require a vast language that most newcomers would shun as being too complex.  Thus, the SBGN community followed an idea used in the development of other standards, i.e. stratify language development into levels.

A \emph{level} of one of the SBGN languages represents a set of features deemed to fit together cohesively, constituting a usable set of functionality that the user community agrees is sufficient for a reasonable set of tasks and goals.  Within \emph{levels}, \emph{versions} represent small evolution of a language, that may involve new glyphs, refined semantics, but no fundamental change of the way maps are to be generated and interpreted. Capabilities and features that cannot be agreed upon and are judged insufficiently critical to require inclusion in a given level, are postponed to a higher level or version.  In this way, the development of SBGN languages is envisioned to proceed in stages, with each higher levels adding richness compared to the levels below it.