\color{red}
\section{Semantic description of \ER{} diagrams}

% Add small images of the glyphs

\subsection{Statements}

An \glyph{interaction} linking the \glyph{interactors} $A$ and $B$ means: ``$A$ interacts with $B$''. An \glyph{outcome} on an \glyph{interaction} represents the cases when the statement is true, that is when the interaction effectively exists. If the interaction is a physical interaction between molecules, the \glyph{outcome} represents the complex resulting from the interaction. It is used as follow: ``when (or if) $A$ interacts with $B$ then \ldots''.\\[\baselineskip]

\noindent
An \glyph{assignement} linking a state variable value $v$ to a \glyph{state-variable} $V$ of an \glyph{interactor} $I$ means: ``$v$ is assigned to $V$ of $I$'' or ``$V$ of $I$ takes the value $v$''. An\glyph{outcome} on an \glyph{assignment} represents the cases when the statement is true, that is when the variable effectively displays the value. It is used as follows: ``when (or if) $V$ of $I$ takes the value $v$ then \ldots'' or more succintly ``when (or if) $I\{V => v\}$ then \ldots''.\\[\baselineskip]

\subsection{Influences}

A \glyph{modulation} linking an \glyph{interactor} $I$ and a relationship $R$ means: ``If $I$ then $R$ is either reinforced or weakened''. An \glyph{outcome} on a \glyph{modulation} represents the cases when the modulation effectively takes place. It is used as follow: ``when (or if) $I$ modulates $R$ then \ldots''.\\[\baselineskip]

% example: Conflicting papers

\noindent
A \glyph{stimulation} linking an \glyph{interactor} $I$ and a relationship $R$ means: ``If $I$ then $R$ is reinforced'' or ``If $I$ then the probability of $R$ is increased''. An \glyph{outcome} on a \glyph{stimulation} represents the cases when the stimulation effectively takes place. It is used as follow: ``when (or if) $I$ stimulates $R$ then \ldots''.\\[\baselineskip]

\noindent
A \glyph{necessary stimulation} linking an \glyph{interactor} $I$ and a relationship $R$ means: ``$R$ only takes place is $I$. An \glyph{outcome} on a \glyph{necessary stimulation} represents the cases when the stimulation effectively takes place. It is used as follow: ``when (or if) $I$ allows $R$ then \ldots''.\\[\baselineskip]

\noindent
An \glyph{inhibition} linking an \glyph{interactor} $I$ and a relationship $R$ means: ``If $I$ then $R$ is weakened'' or ``If $I$ then the probability of $R$ is lowered''. An \glyph{outcome} on an \glyph{inhibition} represents the cases when the inhibition effectively takes place. It is used as follow: ``when (or if) $I$ inhibits $R$ then \ldots''.\\[\baselineskip]

\noindent
An \glyph{absolute inhibition} linking an \glyph{interactor} $I$ and a relationship $R$ means: ``If $I$ then $R$ never takes place''. An \glyph{outcome} on an \glyph{absolute inhibition} represents the cases when the absolute inhibition effectively takes place. It is used as follow: ``when (or if) $I$ blocks $R$ then \ldots''.\\[\baselineskip]

\subsection{Logical Operators}

An \glyph{and} means: ``if all the inputs are true then output''.\\[\baselineskip]

\noindent
An \glyph{or} means: ``if any of the input are true then output''.\\[\baselineskip]

\noindent
A \glyph{not} means: ``if the input is not true then output''.\\[\baselineskip]

\noindent
A \glyph{delay} means: ``if the input is true then output but not immediately''.\\[\baselineskip]

% Add a section on truth table and validation of diagrams. Give examples of false ER diagrams. 

\normalcolor

