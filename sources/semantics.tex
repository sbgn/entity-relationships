\section{Semantic description of \ER{} diagrams}

\subsection{Statements}

An \glyph{interaction} linking the \glyph{interactors} $A$ and $B$ means: ``$A$ interacts with $B$''. An \glyph{outcome} on an \glyph{interaction} represents the cases when the statement is true, that is when the interaction effectively exists. If the interaction is a physical interaction between molecules, the \glyph{outcome} represents the complex resulting from the interaction. It is used as follow: ``when (or if) $A$ interacts with $B$ then \ldots''.\\[\baselineskip]

\noindent
An \glyph{assignement} linking a state variable value $v$ to a \glyph{state-variable} $V$ of an \glyph{interactor} $I$ means: ``$v$ is assigned to $V$ of $I$'' or ``$V$ of $I$ takes the value $v$''. An\glyph{outcome} on an \glyph{assignment} represents the cases when the statement is true, that is when the variable effectively displays the value. It is used as follows: ``when (or if) $V$ of $I$ takes the value $v$ then \ldots'' or more succintly ``when (or if) $I\{V => v\}$ then \ldots''.\\[\baselineskip]

\subsection{Influences}

Modulation: If A then R is either reinforced or weakened\\[\baselineskip]

\noindent
Stimulation: if A then R is reinforced\\[\baselineskip]

\noindent
Necessary stimulation: \\[\baselineskip]

\noindent
Inhibition:  If A then R is weakened\\[\baselineskip]

\noindent
Absolute inhibition: if A then non R\\[\baselineskip]

\subsection{Logical Operators}

and: if A and B then\\[\baselineskip]

\noindent
or: if A or B then\\[\baselineskip]

\noindent
not: if not A then \\[\baselineskip]

\noindent
delay\\[\baselineskip]