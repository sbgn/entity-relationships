\chapter{Grammar of Entity Relationships}
\label{chp:grammar}

\section{Overview}

In this chapter, we describe how the glyphs of \SBGNERLone can be combined
to make a valid \ER{} map. To do this, we must at the very least
define what glyphs can be connected to each other. This is called
syntax. Next, we must define rules over and above connection rules,
such as whether duplicate symbols are permitted. In addition, we must define what the notation ``means'' --- how does it represent a body of biological knowledge? This is semantics, and it is essential if a reader is to understand an SBGN diagram without external help, and a writer is to create one that reflects his understanding of a biological system.

In this section we start off by describing the concepts of the
\ER{} notation. Next a detailed description of the syntax is provided
followed by a description of the syntactic rules of the notation.

\section{Concepts}

The SBGN \ER{} language is more than a collection of symbols. It is a
visual language that uses specific abstractions to describe the
biological processes that make up a quantitative model, a signalling pathway or a metabolic network. This abstraction is the semantics of SBGN, and to describe it requires more than a definition
of the symbols and syntax of the language. We first need to define the
abstractions we are using.

SBGN \ERs{} describe biological interactions involving biological entities. An \glyph{interactor} (\sect{interactors}), such as a molecule, influences the behaviour of other \glyph{interactor} via a relationships. 

It may be convenient to think of a SBGN \ERs{} as listing independent rules that decribe influences between interactors. Diagram can then be analysed with ``what if?'' queries. 

\color{red}
\section{Syntax}

The syntax of SBGN \ERs is defined in the form of an incidence matrix. An incidence matrix has arcs as rows and nodes as columns. Each element of the matrix represents the role of an arc in connection to a node. Input (I) means that the arc can begin at that node. Output (O) indicates that the arc can end at that node. Numbers in parenthesis represent the maximum number of arcs of a particular type to have this specific connection role with the node. Empty cells means the arc is not able to connect to the node. For simplicity Logical operators are treated as interactors. In addition, in \ERs, any relationship can have another relationship as output.

\subsection{Interactor Nodes connectivity definition}  
\begin{tabular}{||c|c|c|c|c|c|c|c|c||}
\hline
\hline
\raisebox{20pt}{$Arc \backslash interactors$} 
& \vglyph{entity} 
& \vglyph{perturbation} 
& \vglyph{observable} 
& \vglyph{outcome}
& \vglyph{and}
& \vglyph{or}
& \vglyph{not}
& \vglyph{delay}
\\ \hline 

\glyph{interaction}           & I &   &   & I &   &   &   &   \\ \hline 
\glyph{assignment}            & O &   &   &   &   &   &   &   \\ \hline 
\glyph{modulation}            & I & I &   & I & I & I & I & I \\ \hline 
\glyph{stimulation}           & I & I &   & I & I & I & I & I \\ \hline 
\glyph{necessary stimulation} & I & I &   & I & I & I & I & I \\ \hline 
\glyph{inhibition}            & I & I &   & I & I & I & I & I \\ \hline 
\glyph{absolute inhibition}   & I & I &   & I & I & I & I & I \\ \hline 
\glyph{logic arc}             & I & I &   & I & IO & IO & IO & IO \\ \hline \hline
\end{tabular}

\subsection{Syntactic rules}

The incidence matrix defining the main part of the syntax is too permissive. 
Additional rules allow to make the syntax definition more precise.

\begin{enumerate}
\item
\end{enumerate}  

\normalcolor

\color{red}
\section{Semantic description of \ER{} diagrams}

\subsection{Statements}

An \glyph{interaction} linking the \glyph{interactors} $A$ and $B$ means: ``$A$ interacts with $B$''. An \glyph{outcome} on an \glyph{interaction} represents the cases when the statement is true, that is when the interaction effectively exists. If the interaction is a physical interaction between molecules, the \glyph{outcome} represents the complex resulting from the interaction. It is used as follow: ``when (or if) $A$ interacts with $B$ then \ldots''.\\[\baselineskip]

\noindent
An \glyph{assignement} linking a state variable value $v$ to a \glyph{state-variable} $V$ of an \glyph{interactor} $I$ means: ``$v$ is assigned to $V$ of $I$'' or ``$V$ of $I$ takes the value $v$''. An\glyph{outcome} on an \glyph{assignment} represents the cases when the statement is true, that is when the variable effectively displays the value. It is used as follows: ``when (or if) $V$ of $I$ takes the value $v$ then \ldots'' or more succintly ``when (or if) $I\{V => v\}$ then \ldots''.\\[\baselineskip]

\subsection{Influences}

A \glyph{modulation} linking an \glyph{interactor} $I$ and a relationship $R$ means: ``If $I$ then $R$ is either reinforced or weakened''. An \glyph{outcome} on a \glyph{modulation} represents the cases when the modulation effectively takes place. It is used as follow: ``when (or if) $I$ modulates $R$ then \ldots''.\\[\baselineskip]

\noindent
A \glyph{stimulation} linking an \glyph{interactor} $I$ and a relationship $R$ means: ``If $I$ then $R$ is reinforced'' or ``If $I$ then the probability of $R$ is increased''. An \glyph{outcome} on a \glyph{stimulation} represents the cases when the stimulation effectively takes place. It is used as follow: ``when (or if) $I$ stimulates $R$ then \ldots''.\\[\baselineskip]

\noindent
A \glyph{necessary stimulation} linking an \glyph{interactor} $I$ and a relationship $R$ means: ``$R$ only takes place is $I$. An \glyph{outcome} on a \glyph{necessary stimulation} represents the cases when the stimulation effectively takes place. It is used as follow: ``when (or if) $I$ allows $R$ then \ldots''.\\[\baselineskip]

\noindent
An \glyph{inhibition} linking an \glyph{interactor} $I$ and a relationship $R$ means: ``If $I$ then $R$ is weakened'' or ``If $I$ then the probability of $R$ is lowered''. An \glyph{outcome} on an \glyph{inhibition} represents the cases when the inhibition effectively takes place. It is used as follow: ``when (or if) $I$ inhibits $R$ then \ldots''.\\[\baselineskip]

\noindent
An \glyph{absolute inhibition} linking an \glyph{interactor} $I$ and a relationship $R$ means: ``If $I$ then $R$ never takes place''. An \glyph{outcome} on an \glyph{absolute inhibition} represents the cases when the absolute inhibition effectively takes place. It is used as follow: ``when (or if) $I$ blocks $R$ then \ldots''.\\[\baselineskip]

\subsection{Logical Operators}

An \glyph{and} means: ``if all the inputs are true then output''.\\[\baselineskip]

\noindent
An \glyph{or} means: ``if any of the input are true then output''.\\[\baselineskip]

\noindent
A \glyph{not} means: ``if the input is not true then output''.\\[\baselineskip]

\noindent
A \glyph{delay} means: ``if the input is true then output but not immediately''.\\[\baselineskip]
\normalcolor
